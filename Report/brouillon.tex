\PassOptionsToPackage{unicode=true}{hyperref} % options for packages loaded elsewhere
\PassOptionsToPackage{hyphens}{url}
\PassOptionsToPackage{dvipsnames,svgnames*,x11names*}{xcolor}
%
\documentclass[man,floatsintext]{apa6}
\usepackage{lmodern}
\usepackage{amssymb,amsmath}
\usepackage{ifxetex,ifluatex}
\usepackage{fixltx2e} % provides \textsubscript
\ifnum 0\ifxetex 1\fi\ifluatex 1\fi=0 % if pdftex
  \usepackage[T1]{fontenc}
  \usepackage[utf8]{inputenc}
  \usepackage{textcomp} % provides euro and other symbols
\else % if luatex or xelatex
  \usepackage{unicode-math}
  \defaultfontfeatures{Ligatures=TeX,Scale=MatchLowercase}
\fi
% use upquote if available, for straight quotes in verbatim environments
\IfFileExists{upquote.sty}{\usepackage{upquote}}{}
% use microtype if available
\IfFileExists{microtype.sty}{%
\usepackage[]{microtype}
\UseMicrotypeSet[protrusion]{basicmath} % disable protrusion for tt fonts
}{}
\IfFileExists{parskip.sty}{%
\usepackage{parskip}
}{% else
\setlength{\parindent}{0pt}
\setlength{\parskip}{6pt plus 2pt minus 1pt}
}
\usepackage{xcolor}
\usepackage{hyperref}
\hypersetup{
            pdftitle={brouillon},
            pdfauthor={Ismail Guennouni~\& Maarten Speekenbrink},
            colorlinks=true,
            linkcolor=blue,
            filecolor=Maroon,
            citecolor=Blue,
            urlcolor=Blue,
            breaklinks=true}
\urlstyle{same}  % don't use monospace font for urls
\usepackage{graphicx,grffile}
\makeatletter
\def\maxwidth{\ifdim\Gin@nat@width>\linewidth\linewidth\else\Gin@nat@width\fi}
\def\maxheight{\ifdim\Gin@nat@height>\textheight\textheight\else\Gin@nat@height\fi}
\makeatother
% Scale images if necessary, so that they will not overflow the page
% margins by default, and it is still possible to overwrite the defaults
% using explicit options in \includegraphics[width, height, ...]{}
\setkeys{Gin}{width=\maxwidth,height=\maxheight,keepaspectratio}
\setlength{\emergencystretch}{3em}  % prevent overfull lines
\providecommand{\tightlist}{%
  \setlength{\itemsep}{0pt}\setlength{\parskip}{0pt}}
\setcounter{secnumdepth}{0}
% Redefines (sub)paragraphs to behave more like sections
\ifx\paragraph\undefined\else
\let\oldparagraph\paragraph
\renewcommand{\paragraph}[1]{\oldparagraph{#1}\mbox{}}
\fi
\ifx\subparagraph\undefined\else
\let\oldsubparagraph\subparagraph
\renewcommand{\subparagraph}[1]{\oldsubparagraph{#1}\mbox{}}
\fi

% set default figure placement to htbp
\makeatletter
\def\fps@figure{htbp}
\makeatother

\shorttitle{Draft-3}
\affiliation{
\vspace{0.5cm}
\textsuperscript{1} Department of Experimental Psychology, University College London}
\usepackage{csquotes}
\usepackage{upgreek}
\captionsetup{font=singlespacing,justification=justified}

\usepackage{longtable}
\usepackage{lscape}
\usepackage{multirow}
\usepackage{tabularx}
\usepackage[flushleft]{threeparttable}
\usepackage{threeparttablex}

\newenvironment{lltable}{\begin{landscape}\begin{center}\begin{ThreePartTable}}{\end{ThreePartTable}\end{center}\end{landscape}}

\makeatletter
\newcommand\LastLTentrywidth{1em}
\newlength\longtablewidth
\setlength{\longtablewidth}{1in}
\newcommand{\getlongtablewidth}{\begingroup \ifcsname LT@\roman{LT@tables}\endcsname \global\longtablewidth=0pt \renewcommand{\LT@entry}[2]{\global\advance\longtablewidth by ##2\relax\gdef\LastLTentrywidth{##2}}\@nameuse{LT@\roman{LT@tables}} \fi \endgroup}


\usepackage{lineno}

\linenumbers

\title{brouillon}
\author{Ismail Guennouni\textsuperscript{1}~\& Maarten Speekenbrink\textsuperscript{1}}
\date{}

\authornote{

Enter author note here.

Correspondence concerning this article should be addressed to Ismail Guennouni, Department of Experimental Psychology, University College London, 26 Bedford Way, London WC1H 0AP, United Kingdom. E-mail: \href{mailto:i.guennouni.17@ucl.ac.uk}{\nolinkurl{i.guennouni.17@ucl.ac.uk}}}

\abstract{
Enter abstract here. Each new line herein must be indented, like this line.


}

\begin{document}
\maketitle

\hypertarget{introduction}{%
\section{Introduction}\label{introduction}}

Being able to transfer previously acquired knowledge to a new domain is one of the hallmarks of human intelligence. Humans are naturally endowed with the ability to extract relevant features from a situation, identify the presence of these features in a novel setting and use previously acquired knowledge to adapt to previously unseen challenges using acquired knowledge. More formally, Perkins (1992) defines transfer of learning as the application of skills, knowledge, and/or attitudes that were learned in one situation to another learning situation. This typically human skill has so far eluded modern AI agents. Deep neural networks for instance can do very well on image recognition tasks and can even reach super-human performance levels on video and strategic board games. Yet they struggle to learn as fast or as efficiently as humans do, and more importantly they have a very limited ability to generalize and transfer knowledge to new domains. (Lake, Ullman, Tenenbaum, \& Gershman, 2017) argue that human learning transfer abilities take advantage of important cognitive building blocks such as an abstract representation of concepts underlying tasks and compositionally structured causal models of the environment.

One way to build abstract representations of the environment when the task involves interactions with others is to build a model of the person we are interacting with that may inform what actions they are likely to take next. Once we learn something about them, we can use this knowledge to inform how to best behave in novel situations. This may lead to very efficient generalization of knowledge, even to situations that are dissimilar to the history of interaction, assuming what we have learned about others is an abstract representation that is not too dependent on the environment of the initial interaction. There is evidence that people learn models of their opponents when they play repeated economic games (Stahl \& Wilson, 1995), engage in bilateral negotiations (Baarslag, Hendrikx, Hindriks, \& Jonker, 2016), or simply try to exploit a non random player in chance games such as Rock-Paper-Scissors (Weerd, Verbrugge, \& Verheij, 2012). In this paper, we are specifically interested in the way in which people build and use models of their opponent to facilitate learning transfer, when engaged in situations involving an interaction with strategic considerations. These situations arise frequently such as in negotiations, auctions, strategic planning and all other domains in which theory of mind abilities (Woodruff, Sciences, Brain, \& 1978, n.d.) play a role in determining human behaviour. In order to explore learning transfer in these strategic settings, it is generally useful to study simple games as a model of more complex interactions. More specifically, we need a framework that allows the study of whether and how a player takes into consideration, over time, the impact of its current and future actions on the future actions of the opponent and the future cumulative rewards. Repeated games, in which players interact repeatedly with the same opponent and have the ability to learn about the opponent's strategies and preferences (Mertens, 1990) are particularly adapted to the task of opponent modelling.

Early literature on learning transfer in games has mostly focused on measuring the proportion of people who play normatively optimal (Nash Equilibria) or salient actions (e.g Risk Dominance) in later games, having had experience with a similar game environment previously. For instance, Ho, Camerer, \& Weigelt (1998) measure transfer as the proportion of players who choose the Nash Equilibrium in later p-beauty contest games, after training on similar games. They find there is no evidence of immediate transfer (Nash equilibrium play in the first round of the new game) but positive structural learning transfer as shown by the faster convergence to equilibrium play by experienced vs non experienced players. ({\textbf{???}}) test learning transfer in players exposed to two games with multiple equilibria sequentially and explore the ability of players to coordinate their actions to choose a particular equilibrium in subsequent games having reached it in prior ones. They distinguish between games that are similar in a purely descriptive way, meaning similar choice labels, identity of players, format and number of action choices; and games that are similar in a strategic sense, meaning similar payoffs from combination of actions, identical equilibrium properties or significant social characteristics of payoffs such as possibility of punishment, need for fairness and cooperative vs competitive settings. They find that transfer of learning (successful coordination) occurs more readily in the presence of both descriptive and strategic similarity. If the games were only strategically similar, then the transfer was much weaker.

Juvina, Saleem, Martin, Gonzalez, \& Lebiere (2013) made a similar distinction between what they deemed surface and deep similarities and find that both contribute to positive learning transfer. However, they show that surface similarity is not necessary for deep transfer and can either aid or block this type of transfer depending on whether it leads to congruent or incongruent actions in later games. In a series of experiments using economic signalling games (Cooper \& Kagel, 2003, 2008) the researchers found that participants who have learned to play according to Nash Equilibrium in one game can transfer this to subsequent games, even though the actions consistent with Nash Equilibrium in later games are different. They show that this transfer is driven by the emergence of sophisticated players who are able to represent the strategic implications of their actions and reason about the consequences of changed opponent payoffs.

\hypertarget{references}{%
\section{References}\label{references}}

\begingroup
\setlength{\parindent}{-0.5in}
\setlength{\leftskip}{0.5in}

\hypertarget{refs}{}
\leavevmode\hypertarget{ref-baarslag2016learning}{}%
Baarslag, T., Hendrikx, M. J. C., Hindriks, K. V., \& Jonker, C. M. (2016). Learning about the opponent in automated bilateral negotiation: a comprehensive survey of opponent modeling techniques. \emph{Autonomous Agents and Multi-Agent Systems}, \emph{30}(5), 849--898.

\leavevmode\hypertarget{ref-cooper2003lessons}{}%
Cooper, D. J., \& Kagel, J. H. (2003). Lessons learned: generalizing learning across games. \emph{American Economic Review}, \emph{93}(2), 202--207.

\leavevmode\hypertarget{ref-Cooper2008}{}%
Cooper, D. J., \& Kagel, J. H. (2008). Learning and transfer in signaling games. \emph{Economic Theory}, \emph{34}(3), 415--439. \url{https://doi.org/10.1007/s00199-006-0192-5}

\leavevmode\hypertarget{ref-ho1998iterated}{}%
Ho, T.-H., Camerer, C., \& Weigelt, K. (1998). Iterated dominance and iterated best response in experimental" p-beauty contests". \emph{The American Economic Review}, \emph{88}(4), 947--969.

\leavevmode\hypertarget{ref-Juvina2013}{}%
Juvina, I., Saleem, M., Martin, J. M., Gonzalez, C., \& Lebiere, C. (2013). Reciprocal trust mediates deep transfer of learning between games of strategic interaction. \emph{Organizational Behavior and Human Decision Processes}, \emph{120}(2), 206--215. \url{https://doi.org/10.1016/j.obhdp.2012.09.004}

\leavevmode\hypertarget{ref-Lake2017}{}%
Lake, B. M., Ullman, T. D., Tenenbaum, J. B., \& Gershman, S. J. (2017). Building machines that learn and think like people. \emph{Behavioral and Brain Sciences}, \emph{40}. \url{https://doi.org/10.1017/S0140525X16001837}

\leavevmode\hypertarget{ref-mertens1990repeated}{}%
Mertens, J.-F. (1990). Repeated games. In \emph{Game theory and applications} (pp. 77--130). Elsevier.

\leavevmode\hypertarget{ref-stahl1995players}{}%
Stahl, D. O., \& Wilson, P. W. (1995). On players models of other players: Theory and experimental evidence. \emph{Games and Economic Behavior}, \emph{10}(1), 218--254.

\leavevmode\hypertarget{ref-de2012higher}{}%
Weerd, H. de, Verbrugge, R., \& Verheij, B. (2012). Higher-order social cognition in rock-paper-scissors: A simulation study. \emph{Proceedings of the 11th international conference on autonomous agents and multiagent systems-volume 3}, 1195--1196.

\leavevmode\hypertarget{ref-Woodruff}{}%
Woodruff, G., Sciences, D. P. B., Brain, \& 1978. (n.d.). \emph{Does the chimpanzee have a theory of mind}.

\leavevmode\hypertarget{ref-baarslag2016learning}{}%
Baarslag, T., Hendrikx, M. J. C., Hindriks, K. V., \& Jonker, C. M. (2016). Learning about the opponent in automated bilateral negotiation: a comprehensive survey of opponent modeling techniques. \emph{Autonomous Agents and Multi-Agent Systems}, \emph{30}(5), 849--898.

\leavevmode\hypertarget{ref-cooper2003lessons}{}%
Cooper, D. J., \& Kagel, J. H. (2003). Lessons learned: generalizing learning across games. \emph{American Economic Review}, \emph{93}(2), 202--207.

\leavevmode\hypertarget{ref-Cooper2008}{}%
Cooper, D. J., \& Kagel, J. H. (2008). Learning and transfer in signaling games. \emph{Economic Theory}, \emph{34}(3), 415--439. \url{https://doi.org/10.1007/s00199-006-0192-5}

\leavevmode\hypertarget{ref-ho1998iterated}{}%
Ho, T.-H., Camerer, C., \& Weigelt, K. (1998). Iterated dominance and iterated best response in experimental" p-beauty contests". \emph{The American Economic Review}, \emph{88}(4), 947--969.

\leavevmode\hypertarget{ref-Juvina2013}{}%
Juvina, I., Saleem, M., Martin, J. M., Gonzalez, C., \& Lebiere, C. (2013). Reciprocal trust mediates deep transfer of learning between games of strategic interaction. \emph{Organizational Behavior and Human Decision Processes}, \emph{120}(2), 206--215. \url{https://doi.org/10.1016/j.obhdp.2012.09.004}

\leavevmode\hypertarget{ref-Lake2017}{}%
Lake, B. M., Ullman, T. D., Tenenbaum, J. B., \& Gershman, S. J. (2017). Building machines that learn and think like people. \emph{Behavioral and Brain Sciences}, \emph{40}. \url{https://doi.org/10.1017/S0140525X16001837}

\leavevmode\hypertarget{ref-mertens1990repeated}{}%
Mertens, J.-F. (1990). Repeated games. In \emph{Game theory and applications} (pp. 77--130). Elsevier.

\leavevmode\hypertarget{ref-stahl1995players}{}%
Stahl, D. O., \& Wilson, P. W. (1995). On players models of other players: Theory and experimental evidence. \emph{Games and Economic Behavior}, \emph{10}(1), 218--254.

\leavevmode\hypertarget{ref-de2012higher}{}%
Weerd, H. de, Verbrugge, R., \& Verheij, B. (2012). Higher-order social cognition in rock-paper-scissors: A simulation study. \emph{Proceedings of the 11th international conference on autonomous agents and multiagent systems-volume 3}, 1195--1196.

\leavevmode\hypertarget{ref-Woodruff}{}%
Woodruff, G., Sciences, D. P. B., Brain, \& 1978. (n.d.). \emph{Does the chimpanzee have a theory of mind}.

\endgroup

\end{document}
