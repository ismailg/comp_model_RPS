\documentclass[12pt]{article}         % the type of document and font size (default 10pt)
\usepackage[margin=1.0in]{geometry}   % sets all margins to 1in, can be changed
\usepackage{moreverb}                 % for verbatimtabinput -- LaTeX environment
\usepackage{url}                      % for \url{} command
\usepackage{amssymb}                  % for many mathematical symbols
\usepackage[pdftex]{lscape}           % for landscaped tables
\usepackage{longtable}                % for tables that break over multiple pages
\title{Transfer of Learned Opponent Models in Zero Sum Games}  % to specify title
\author{Ismail Guennouni, Maarten Speekenbrink}             % to specify author(s)
\usepackage{Sweave}
\begin{document}                      % document begins here

% If .nw file contains graphs: To specify that EPS/PDF graph files are to be 
% saved to 'graphics' sub-folder
%     NOTE: 'graphics' sub-folder must exist prior to Sweave step
%\SweaveOpts{prefix.string=graphics/plot}

% If .nw file contains graphs: to modify (shrink/enlarge} size of graphics 
% file inserted
%         NOTE: can be specified/modified before any graph chunk
\setkeys{Gin}{width=1.0\textwidth}

\maketitle              % makes the title
%\tableofcontents        % inserts TOC (section, sub-section, etc numbers and titles)
%\listoftables           % inserts LOT (numbers and captions)
%\listoffigures          % inserts LOF (numbers and captions)
%                        %     NOTE: graph chunk must be wrapped with \begin{figure}, 
%                        %  \end{figure}, and \caption{}
%%%%%%%%%%%%%%%%%%%%%%%%%%%%%%%%%%%%%%%%%%%%%%%%%%%%%%%%%%%%%%%%%%%%
% Where everything else goes


\Sconcordance{concordance:draft_report.tex:draft_report.Rnw:%
1 9 1 1 0 51 1}


\section*{Introduction}

Being able to transfer previously acquired knowledge to a new domain is one of the hallmarks of human intelligence (Taylor et al., 2008). Humans are naturally endowed with the ability to extract relevant features from a situation, identify the presence of these features in a novel setting and use previously acquired knowledge to adapt to previously unseen challenges using acquired knowledge. More formally, Perkins (1992) defines transfer of learning as the application of skills, knowledge, and/or attitudes that were learned in one situation to another learning situation. This typically human skill has so far eluded modern ai agents. Deep neural networks for instance can do very well on image recognition tasks and can even reach super-human performance levels on video and strategic board games. Yet they struggle to learn as fast or as efficiently as humans do, and more importantly they have a very limited ability to generalize and transfer knowledge to new domains. Lake et al.(2017) argue that human learning transfer abilites take advantage of important cognitive building blocks such as a deeper representation of concepts underlying tasks and compositionally structured causal models of the environment.


In this paper, we are specifically interested in the way in which people build and use models of their opponent to facilitate learning transfer, when engaged in situations involving an interaction with strategic considerations. These situations arise frequently such as in negotiations, auctions, strategic planning and all other domains in which theory of mind abilities (Premack & Woodruff, 1978) play a role in determining human behaviour.  

In order to explore learning transfer in strategic settings, it is generally useful to study simple games as a model of more complex interactions. More specifically, when studying learning, we need a framework that allows the study of whether and how a player takes into consideration, over time, the impact of its current and future actions on the future actions of the opponent and the future cumulative rewards. Repeated games, in which players interact repeatedly with the same opponent and have the ability to learn about the opponent's strategies and preferences (Mertens, 1990) are particularly adapted to modelling learning about the opponent.

Early literature on learning transfer in games has mostly focused on measuring the proportion of people who play normatively optimal (Nash Equilibria) or salient actions (e.g Risk Dominance) in later games, having had experience with a similar game environment previously. For instance, Ho et al. (1998) measure transfer as the proportion of players who choose the Nash Equilibrium in later p-beauty contest games, after traingn on similar games. They find there is no evidence of immediate transfer (Nash equilibrium play in the first round of the new game) but positive structural learning transfer as shown by the faster convergence to equilibrium play by experienced vs non experienced players. Camerer & Knez (2000) test learning transfer in players exposed to two games with multiple equilibria sequentially and explore the ability of players to conordinate their actions to choose a particular equilibrium in subsequent games having reached it in prior ones. They distinguish between games that are similar in a purely descriptive way, meaning similar choice labels, identity of players, format and number of action choices; and games that are similar in a strategic sense, meaning similar payoffs from combination of actions, identical equilibrium properties or significant social characteristics of payoffs such as possibility of punishment, need for fairness and cooperative vs competitive settings. They find that transfer of learning (successful coordination) occurs more readily in the presence of both descriptive and strategic similarity. If the games were only strategically similar, then the transfer was much weaker.  


Juvina et al.(2014)  made a similar distinction between what they deemed surface and deep similarities and find that both contribute to positive learning transfer. However, they show that surface similarity is not necessary for deep transfer, and can either aid or block this type of transfer depending on whether it leads to congruent or incongruent actions in later games. In a series of experiments using economic signalling games, Cooper & Kagel (2003, 2004, 2008) find that participants who have learned to play according to Nash Equilibrium in one game can transfer this to subsequent games, even though the actions consistent with Nash Equilibrium in later games are different.  They show that this transfer is driven by the emergence of sophisticated players who are able to represent the strategic implications of their actions and reason about the consequences of changed opponent payoffs.  

Most of these studies fail to offer a formal explanation of this transfer or a modelling framework that can explain the experimental observation of transfer between games and generalise it to extensive classes of games. A notable exception is the effort by Haruvy and Stahl (2012) to specify a model of learning where players learn abstract rules that they can generalise and transfer across dissimilar games, rather than action choices that can only be used within the same game. Participants played ten games, presented as 4x4 normal form (matrix payoffs). Their results suggest that subjects do transfer learning over descriptively similar but strategically dissimilar games and that this learning transfer is significant. They also showed that players learn abstract aspects of the game that are then transfered to new settings. Their rule learning model, based on Stahl (1996) was able to capture participants dynamic behavior and show that the propensity to select particular rules is perfectly transfered across games.  
 
In most of the experimental paradigms we saw, participants were playing other human players either one on one or as a group. In this case, players are learning about the game but so are their opponents. While the approach of matching groups of human players repeatedly to play economic games has ecological validity in recreating environments where social decision making can be studied, it complicates the detection and modelling of strategies used by participants. It is harder to focus on an individual and how her strategies are changing and adapting to the opponent's play if we cannot experimentally control the behaviour of the opponent. It is also difficult to disentangle the process of learning about the opponent from that of learning about the game structure and payoffs. One potential solution is to replace the human opponent with a computerised agent, allowing researchers to manipulate the opponent's behaviour and observe how the human participants adapt their behaviour. For instance, Spiliopoulos (2013) made humans play constant sum games against 3 computer opponents programmed to take advantage of known patterns in human play such as imperfect randomization and heuristics use, and found that human participants do adapt to the opponent they are facing. Shachat & Swarthout (2004) made human participants face computer opponents playing various mixed strategies in a zero-sum asymmetric matching pennies game. They found that the players changed their strategies towards exploiting the deviations from the MSNE, and that this exploitation was very likely if the deviation from the MSNE play was high.


One of the major limitations in studies of how humans adapt to computerised opponents is that they have mostly looked at the ability of players to detect and exploit action-based learning rules. They either used deterministic strategies that did not consider the human participant's play, as in the case of pre-determined mixed strategy algorithms, or strategies that only depended on the player's patterns of past plays. These strategies did not necessarily take into account the computer's own past plays, or the degree of strategic reasoning of the human player about the computer. Their play was not "human-like" in the sense that humans are not very good at playing specific mixed strategies with any precision of at detecting patterns from long sequences of past play due to cognitive constraints. It is therefore important to have agents that "play like humans", and one way of achieving that is to embed theses agents with human-like theory of mind abilities based on limited steps of recursive reasoning. Simon (1972) explains that humans have limited cognitive capacities and as such cannot be expected to solve computationally intractable problems such as finding Nash equilibria. Instead, they will try to 'satisfice' by choosing a strategy that is adequate in a simplified model of the environment, rather than an optimal one. This concept finds its natural application in 'level-k' theory, first adopted by Stahl & wilson (1995). It posits that deviations from Nash equilibrium solutions in simple games are explained by the fact that humans have a heterogenous degree of strategic sophistication. At the bottom of the ladder, level-0 players are non-strategic and play either randomly or use a salient strategy in the game environment (Arad & Rubinstein, 2012). Level-1 players are next up the ladder of strategic sophistication and will assume all their opponents belong to the level-0 category and as such will best respond to them given this assumption. Likewise, a level-2 player will choose actions that are the best response given the belief that all opponents are exactly one level below and so on. 

Another potential issue in the experimental paradigms used in the learning transfer literature is the use of social dilemmas and cooperative/competitive settings that may invoke confounding factors in learning and transfer. Social dilemma and coordination games crystallise the conflict between competing with others out of self-interest or cooperating to reach a socially optimal outcome. Playing these games repeatedly implicates important social constructs such as reputation building, trust and other individual psychological attributes such as cooperativeness and inequity aversion. Coordination games also test the ability of choosing the safe self-interested choice compared to the risky cooperative choice and may depend on similar latent factors. As such, these confounding factors may impede teasing out the process of learning an opponent model and that of whether/how these models are transfered. By contrast, purely competitive settings and zero sum games in particular would be more appropritate. They do not incentivize any cooperation, since one player's gain is necessarily the opponent's loss, and are agnostic to trust mechanisms or reputation building. Also choosing games with no pure strategy Nash Equilibrium shift the focus away from learning a normatively optimal action to that of reasoning dynamically about the opponent which is more amenable to modelling learing. Games with unique mixed strategy nash equilibrium leading to nil average rewards facilites infering whether participants have learned to exploit the non random play of the opponent.
  

To address some of these limitations in the exisintg literature on learning transfer outlined above, we propose to explore opponent modelling and its transfer with the use of computer agents posssessing human-like theory of mind abilities with limited degrees of iterated reasoning. The agents will have a fixed strategy played stochastically, and will either be level-1 or level-2 behavioral rule, mimicking human theory of mind abilities and the limited recursion depth they exhibit (Goodie et al., 2012). We also use a set of purely competitive zero-sum games that incentivise reasoning about the opponent without introducing social confounding factors. We measure transfer of learning about the opponent strategy between games that are similar and dissimilar. The first two games we use are identical except for action labels. In one experiment, the third game is strategically similar to the first two but descriptively different, while in a second experiment, we introduce a third game that is dissimilar to the first two in terms of payoff matrix and strategic structure while retaining, like all other games, a unique mixed strategy Nash equilibrium of random action with 0 expected reward against a Nash player. 

Finally, unlike the vast majority of experimental paradigms, we make participants play the games in an engaging, interactive and ultimately more ecological way rather than simply provide the matrix form of the game. We use visual aids representing the player and their computer opponent as well as the various actions that people can select built on a highly interactive interface providing image, audio and sometimes video feedback at each step, from choosing actions, to presenting outcomes of the rounds and players scores. 



\section*{Methods}

We ran two experiments where human participants played against computer opponents a total of 3 different games. In the first experiment, participants played 3 games sequentially against the same computer opponent. The computer opponent either used a level-1 or level-2 strategy. The three games were rock-paper-scissors, fire-water-grass and the numbers game.  A typical rock-paper-scissors game (hereafter RPS) is a 3x3 zero sum game, with a cyclical hierarchy between possible actions: rock blunts scissors, paper wraps rock, and scissors cut paper. If one player choses an action which dominates their opponent's action, the player wins (receives a reward of 1) and the other player loses (receives a reward of -1). Otherwise it is a draw and both players receive a reward of 0. It has a unique MSNE consisting of randomly playing one of the three options each time.  

The second game is identical to Rock-Paper-Scissors in all but action labels. We call it Fire-Water-Grass (FWG): Fire burns grass, water extinguishes fire, and grass absorbs water. We are interested in exploring whether learning is transferred in a fundamentally similar game where the only difference is in the description of the choice actions.  Finally, the numbers game is a generalization of rock-paper-scissors. In the variant we use, 2 participants concurrently pick a number between 1 and 5. To win in this game, a participant needs to pick a number exactly 1 higher than the number chosen by the opponent. For example, if a participant thinks their opponent will pick 3, they ought to choose 4 to win the round. To make the strategies cyclical as in RPS, the game stipulates that the lowest number (1) beats the highest number (5), so if the participant thinks the opponent will play 5, then the winning choice is to pick 1. This game has a structure similar to RPS in which every action is dominated by exactly one choice. All other possible combination of choices that are not consecutive are considered ties. A win would add 1 point to the score of the player, while a loss deduces one point and a tie does not affect the score. Similar to RPS, the MSNE is to play each action with equal probability in a random way.  

Participants were informed they would play three different games against the same computer opponent. Each participant plays all three games consecutively and in the same order described above. Participants were told that the opponent cannot cheat and will choose its actions simultaneously without knowledge of the participant's choice. A total of 50 rounds of each game was played with the player's score displayed at the end of each game. The score was calculated as the number of wins minus the number of losses. Ties did not affect the score. In order to incentivise the participants to maximise the number of wins against the opponents, players were paid a bonus at the end of the experiment that was proportional to their final score. The overall score of the players was translated into the bonus by making each point worth $£$0.02. This bonus is significant as players could increase the total payoff from the experiment by up to 60\% assuming they'd won all rounds against the computer opponent.


In the second experiment, participants each played 3 games sequentially against both level-1 and level-2 computer opponents, rather than just one like in the first experiment. The three games were Rock-Paper-Scissors, Fire-Water-Grass, and the penalty shootout game. The first two games were identical to the ones used in the first experiment. In the final game (shootout) the participants played the role of the player shooting a football (soccer) penalty, and the AI opponent was the goalkeeper. Players had the choice between three actions, like in the first two games: Shooting the football to the left, right or centre of the goal. If the player shoots in a direction different from that of where the goalkeeper dives, they win the round and the goalkeeper loses. Otherwise, the goalkeeper catches the ball and the player loses the round. There is no possibility of ties in this game. What makes this game different however is that there are two ways to beat the opponent in each round: if we think the opponent is going to choose ''right'' in the next round, we can win by both choosing ''left'' and ''center''. A level-1 player (thinks that his opponent will repeat his last action) has two ways to win the next round. As such, we have programmed the level-2 computer program to choose randomly between the two possibilities that a level-1 player may choose.  

Like in the first experiment, the computer opponents retained the same strategy throughout the 3 games, however the participants faced each opponent twice in each game. Specifically, each game was divided into 4 stages numbered 1 to 4, consisting of 20, 20, 10, and 10 rounds respectively for a total of 60 rounds per game. Participants start by facing one of the opoonents in stage one, then face the other in stage two. This is repeated in the same order in stages 3 and 4. Which opponent they faced first was counterbalanced. All participants engage in the same three games (namely RPS, FWG and Shootout) in this exact order, and were aware that the opponent was not able to know their choices beforehand but was choosing simultaneously with the player. 

In order to encourage participants to think about their next choice, a countdown timer of 3 seconds was introduced at the beginning of each round. During those 3 seconds, participants could not choose an option and had to wait for the timer to run out. A small delay that changed randomly (between 0.5 and 3 seconds) was also introduced in the time it took the AI agent to give back their response to make it seem like it is also thinking about its last actions, as a way of simulating a real human opponent. After each round, the participants were given detailed feedback about their opponent actions as well as whether they won or lost the last round. Further information about the outcome of previous rounds was also visible on the game screen below the feedback area, throughout each stage game and opponent could go back many rounds to study the history of interaction. The number of wins, losses and ties were clearly shown at the top of the screen for each game, and this scoreboard was reinitialised to zero at the onset of a new stage game. An example of the interface for the rock-paper-scissors game is provided in figure XXXXXXX 

To summarise, all the games used in both experiments have a unique MSNE consisting of randomising across actions. If participants follow this strategy, or simply dont engage in learning how the opponent plays, they woud score 0 on average against both level-1 and level-2 players. Evidence of sustained wins would indicate that participants have learned to exploit patterns in the opponent play. 



\section*{Results}



\section*{Discussion}



\section*{Conclusion}

  
\section{References}
Goodie, A. S., Doshi, P., & Young, D. L. (2012). Levels of theory-of-mind reasoning in competitive games. Journal of Behavioral Decision Making, 25(1), 95–108.
Return to ref 34 in article

\end{document}
